\documentclass[11pt]{report}

\usepackage{ucs}
\usepackage[utf8x]{inputenc}
\usepackage{titlesec}
\usepackage{graphicx}
\usepackage{amssymb}
\usepackage{amsmath}
\usepackage{dsfont}
\usepackage{caption}
\usepackage{subcaption}
\usepackage{array}



\title{\textbf{TS114 Project}\\Computer-aided analysis of electrocardiogram signals}
\author{{Maxime PETERLIN - Gabriel VERMEULEN }\\\\{ENSEIRB-MATMECA, Bordeaux}}
\date{June, 6th 2014}


\begin{document}


\maketitle

\tableofcontents

\newpage
\section*{Introduction}


\chapter{ECG visualization}
	In this first part, seven different ECG signals will be analyzed and displayed under MATLAB. \\Three of these are records from healthy patient's heart, whereas the other four are from ill ones, each with different pathologies.\\
	This part's aim is to study these signals in the frequency and time domain, starting by the latter, and underline the different caracteristics and properties of these.

	\section{Time display}
		Firstly, the aforementioned normal ECG signals (i.e. from healthy heart) were plotted under MATLAB. On the following figure, their Q, R and S points are represented by red dots, while their P and T waves are located by green dots.  

	\section{Frequency display}


\chapter{Detection of P, QRS and T waves}

\chapter{Automatic identification of cardiac pathologies}

\chapter{ECG denoising}


\newpage
\section*{Conclusion}

\end{document}
